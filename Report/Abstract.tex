\begin{abstract}
	Terrain modelling and representation is a elaborated topic in computer graphics and visualization. This report highlights some of the state-of-the-art techniques for large scale modelling of terrains, as well was modelling procedurally generation of content. We take a closer look on feature based modeling approach, where iconic landmarks, such as rivers or hills, can be combined with base terrain features to produce arbitrary, but realistic looking terrains. A large part in terrain generation is the creation of drainage networks and the influence of fluids to the terrain. We examine a procedure, that generates a functioning river network and create a terrain around it. Lastly the influence of water on terrain is a big research field. Here we look deeper into algorithms, that simulate the erosion of terrain due to fluid streams, accounting for both sediment and dissolve of soil material. 
	
	\begin{classification} % according to http://www.acm.org/about/class/2012
		\CCScat{Computing methodologies}{Computer graphics}{Rendering}
	\end{classification}
	
\end{abstract}
