\section{Conclusion and outlook}
We presented many challenges and approaches in the field of procedural terrain generation, as well as many solutions on those topics - many of which can be performed in real-time. There are many other field of research we did not scratch in this report, such as generation of flora and fauna on both large and small scale terrain, complete weather simulation, procedural city generation or agent-based simulation. Real-time applications generate solutions with plausibility in mind, but not physical correctness. Nonetheless such results look stunning most of the time, and can easily be mistaken for real data sets. But this does not mean that any kind of (un-)realistic terrain can be created. Usually the "rules of physical correctness" are weakened and simplified in order to retain real-time capability of such algorithms. Today those applications are mainly used for content creation.

Other applications aim for physical correctness, but they usually cannot be computed in real time. Those more exact simulations rely on very heavy computing, with a physically correct result in mind. As we have seen here, in many cases a combination of several approaches can yield realistic and at least partially correct results at reasonable computation time. But when we look at the recent development and advances in (cheaply) available computing power, we can see the possibilities have tremendously increased along with it. Keeping that in mind, one might consider it quite possible - if not very likely - that physically correct calculations might one day even be possible in real time.