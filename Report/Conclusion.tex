\section{Conclusion and outlook}
We presented many field of procedural generation, as well as many solutions on those topics, many of which can be performed in real-time. There are many other field of research we did not scratch in this report, such as generation of flora and fauna on both large and small scale, weather simulation, procedural city generation or agent-based simulation. Real-time applications generate solutions with plausibility in mind, but not physical correctness. Nonetheless such results are stunningly looking and can be easily mistaken for real data sets. This does not mean, any kind of unrealistic terrains can be created, usually the rules of physical correctness are weakened and simplified in order to achieve real-time performance. Those applications are mainly used for content creation.

Other applications aim for physical correctness, but they usually cannot be computed in real time. Those exact simulations rely on very heavy computing, with an physically correct result in mind. As we have seen, in many cases a combination of several approaches result in realistic and (partly) correct systems at reasonable computation time and artistic resources. But when we look at the recent development and advances in (cheaply) available computing power, we see the possibilities have tremendously increased along with it. Keeping that min mind, one might consider it quite possible - if not very likely - that physically correct calculations might one day even be possible in real time.