\section{Heat Transfer Simulation for Modeling Realistic Winter Sceneries}

!!! RAW TEXT !!!

While many of the previously presented papers strictly deal with erosion in combination with evaporation and seminentation alone, the temperature is mostly assumed to be constant. In some simpler cases it is not take into account at all. In paper presented in this chapter, exactly that often „overlooked“ topic becomes the primary focus of the calculations.

To allow more complex thermal calculation for simulating a natural environment, one needs to take a lot of different influences and physical processes into account. The existing methods prior to this paper primarily rely on one of those three approaches for calculating winter scenarios for example:
- Particle based snow accumulation
- Surface displacement methods
- Ice growth
The name of these methods basically already implies what they fundamentally do:
The particle based accumulation methods evaluate the trajectory of snowflakes (blown by the wind) colliding with the surface (). In an alternative approach the stochastically generated snowflakes are „shot up in the sky“ and the calculations are done recursively. Non of these methods are truly physics-based and both either include solving Navier-Stokes () or the Bolzmann equations (). The only physics based system mentioned, works on basis of vortex fields and incorporates melting snow, but leaves the weather completely aside.
The surface displacement methods basically only calculate the hight of the accumulated snow falling on a specific spot. This can easily be done by using the depth buffer () or through dissipating calculated ambient occlusion with illumination (). Another very efficient variant of this approach is done by using hight span maps and heavily relies on a statistical model for snow accumulation derived from observations in nature ().
Calculating ice growth is not used so widely. Some methods simulate ice crystal formation over objects by using a phase field method (). Another more sophisticated hybrid-approach combines those phase filed methods with fluid simulation and procedural defused limited aggregation ().

The newly proposed approach
Like many other methods, this one also relies on a voxel grid. Like for so many other authors before mentioned, the challenge is to find the right balance between a high-resolution grid which is able to capture small details, and a large scale grid which requires less memory and computational effort.
Here this problem is handled by evaluating the evolution of temperature of every voxel, taking into account changing weather and environmental conditions (and therefore temperate) over time. After that a hightfield is generated for the surface representation, from the hight of snow stored in the according voxels ().
The data structure used for the calculation consists of a grid of voxels, where for each voxel one (or two) materials, the temperature and the percentage of solid material are stored along with energy exchange over time. Since some voxels can contain two materials at once (e.g. ice and water or snow and air), these voxels need to be evaluated more carefully. Phase transitions are more likely to occur in those „mixed“ voxels than in any other one.

!!! RAW TEXT !!!