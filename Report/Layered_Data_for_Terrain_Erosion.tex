\section{Layered Data Representation for Visual Simulation of Terrain Erosion}

\cite{marechal2010heat} - !!! RAW TEXT !!!
%DEL: \cite{cordonnier2016large}

This Paper focuses on introducing a completely new data-structure to store terrain-information which makes calculations much easier to perform. The two opposing conventional methods of storing and handling such data have already been explained before:
- Voxel Representations which allow for accurate calculations but have very high memory demands
- Representations of the just the terrain-surface with hightfield(s) where precision and accuracy is sacrificed in order to gain space and reduce computing requirements.

Many different techniques have already been discussed here and many more have been developed. Some of the most common ones include fractal interpolation to generate hills and water streams (---) or blending some elaborate noise functions to generate visually plausible results (---). But all of there approaches have one thing in common: while the accuracy of true 3D Voxel representation is often approximated quite well, is is never truly reached with those techniques.
To overcome this limitation and enable very accurate calculations, which are meant to be used in physical simulations rather than in realtime environments, the authors generated a mixed data-structure combining Voxel and hight filed representations.

\subsection{A novel datastructure}
Since in many areas of a given terrain full Voxel data would be "waste  of data" \cite{marechal2010heat} since the represented layers are usually quite thick, one can reduce the needed space in some areas. This approach used in this paper is to use a 2D array of 1D arrays as representation of terrain. This can be  imaged as a core sample, where each layer in itself is supposed to be constant (or rather is's parameters are). The hight of the position can easily derived from a "core samples" hight and it's index.
The big advantage of this kind of data structure is it’s easy and fast traversal with a much smaller effort (and thus higher speed) as it would be the case for a conventional pure-Voxel approach. Another positive side effect of storing data this was is that the data structure also allows for zero-density layers. This way also caves and even material falling from the ceiling of those caves (along the inverted surface gradient) can be captured and therefore be used to perform calculations. That way erosion can not only be performed on the terrain-surface, but also in all sub-surface holes existing or developing in the terrain \cite{marechal2010heat}.

\subsection{Efficiency}
The calculations to test the efficiency of the newly introduced data structure were performed on an Intel III with 500MHz, and a freeware tool was used to generate simple visualizations. The calculation of 100 erosion steps on a 1024x1024 element grid with 5 layers took 239 minutes for example. Other calculation taking much longer were also tested, including ones, that could not be done with relying on only hight filed data. The full and exact specs and results can be found in the original paper \cite{marechal2010heat}.