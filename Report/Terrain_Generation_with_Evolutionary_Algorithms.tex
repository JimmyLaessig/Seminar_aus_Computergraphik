\section{Procedural Terrain Generation Techniques using Evolutionary Algorithms}

\begin{figure}[htb]
	\centering
	\includegraphics[width=\linewidth]{RZL12/06256610_1.png}
	\caption{Example for terrain generation by fractal subdivision}
	\label{fig:frag_subdev}
\end{figure}

This paper \cite{raffe2012survey} is rather a state of the art report itself than a traditional publication presenting a novel approach or technique for the first time. Nevertheless this paper gives us great insight to a variety of different techniques and approaches we have not been able to discuss here yet. Furthermore the viewpoint from the authors is fundamentally different to most of the other papers which have been mentioned so far.

\begin{figure}[htb]
	\centering
	\includegraphics[width=\linewidth]{RZL12/5zr45zr6z5.png}
	\caption{Example of results with the algorithm from \cite{walsh2011use}}
	\label{fig:tag17}
\end{figure}

This paper completely discards things like physical correctness and/or plausibility, and entirely focuses on algorithms and techniques for creating suitable terrain for game-environments. The first big difference that comes with that focus is the fact, that the only data structure used here are hight maps. Since physical correctness or the process of simulating erosion steps or something like that are considered to be irrelevant in this scenario, it would not make much scene to deal with the before-mentioned downsides of 3D voxel representations.

\subsection{Different types of algorithms}
The applied techniques for generating a feasible terrain for game scenarios can be categorized as two different types:
\begin{itemize}
	\item Fractal Subdivision 
	\item Evolutionary Algorithms
\end{itemize}

Algorithms categorized as the 1st type, have been used for many years already. An illustrative example can be seen in figure \ref{fig:frag_subdev}. The technique is clearly the older one of the two mentioned, but is also well researched and a lot of well established algorithms and variations of it exist. The possible user input is usually controlled through several input-parameters to the formulas used for the calculations. An examples for such parameters would simply be the number of specific terrain features (mountains, cliffs, etc.) \cite{raffe2012survey}. As it can easily be imagined, this makes the amount of possible control through user-interaction during the generation process quite small.

\begin{figure}[htb]
	\centering
	\includegraphics[width=\linewidth]{RZL12/tz76jt7ut.jpg}
	\caption{Example of results with the algorithm from \cite{ong2005terrain} and \cite{saunders2006realistic}}
	\label{fig:tag15}
\end{figure}


In contrast to that, the 2nd type - Evolutionary Algorithms - are a much newer field of research. The approaches rely on the output of a variety of well established techniques for generating random noise or data (like fractals for example). An example terrain generated with the technique of \cite{togelius2010towards} can be seen in figure \ref{fig:tag23}. User interaction can potentially be much greater here, since there are a variety of different fitness functions used in evolutionary algorithms. In addition to that, these techniques also allow for a variety of input parameters. That way evolutionary algorithms usually perform vastly better then fractal subdivision-approaches when it comes to user-interaction. Like the title of this paper suggests, this type of algorithms will be looked upon more closely in the following part of this paper.

\begin{figure}[htb]
	\centering
	\includegraphics[width=\linewidth]{RZL12/hfhhf.jpg}
	\caption{Example of results with the algorithm from \cite{frade2009breeding}}
	\label{fig:tag19}
\end{figure}

\subsection{Evolutionary Algorithms}
In this state of the art report several different papers are described, evaluated and compared. Many of the presented methods have already been incorporated and used in several popular computer games commercially available. One important fact to consider is that not every technique can be used for every game since different games and genres also have different requirements for the terrain to be generated. For a first-person shooter it might not be a big problem if a pattern repeats itself several times on a rather small-scaled map, while it would be a huge obstacle for a computer game where the player should be motivated to explore new areas of the terrain (like an RPG for example) \cite{raffe2012survey}. In both of these cases accessibility - so that most parts of the terrain can be traversed by the player's character - is very important. But if we consider terrain-generation for a flight simulator on the other side, the requirements will be fundamentally different since terrain-traversal is of no concern in this case.

\begin{figure}[htb]
	\centering
	\includegraphics[width=\linewidth]{RZL12/06256610.jpg}
	\caption{Example of results with the algorithm from \cite{frade2010evolution2}}
	\label{fig:tag21}
\end{figure}

\subsubsection{Modifying user-provided samples}
Many of the methods presented in this report do not actually create new terrain, but rather modify or recombine user-provided samples. In one case (\cite{walsh2010terrain} followed by \cite{walsh2011use}) this is done by simply modifying the given terrain by applying / changing different input parameters (see figure \ref{fig:tag17}). This apporach has of cause one big advantage: the resulting terrain is quite sure to fit the same requirements as the input-terrain did. In addition to that the modifications are highly controllable by the user.

In another case \cite{raffe2011evolving} the user-provided sample-terrains (the more the better) are decomposed into many smaller patches, which are then recombined in different positions to a completely new terrain (an example can be seen in figure \ref{fig:tag25}). The possibility for a patch to be relocated and where it is moved to, is stochastically determined for each patch. The quality as well as the plausibility of the resulting terrain is of cause vastly defined by the choosen patch size to work with. One important fact to consider here is, that the patches are usually quadratic / rectangular. Therefore it can be quite difficult - if not impossible - to generate triangular terrain features with this techniques.

An advantage of both those techniques and many other like them, is the simplicity of generating new maps. This can easily be done automatically without big computational effort, as well as from an unexperienced user in a map-editor for example. The big downside however, is the bad exploration of the solution space. Since every feature of a possible result from such techniques must be present in at least one of the user-provided samples, the resulting "new" terrain has very limited features.

\begin{figure}[htb]
	\centering
	\includegraphics[width=\linewidth]{RZL12/rthrhrh.jpg}
	\caption{Example of results with the algorithm from \cite{togelius2010towards}}
	\label{fig:tag23}
\end{figure}

\subsubsection{Generating terrain from scratch}
Of cause there also are some techniques aiming to generate completely new terrains from scratch. For example in the technique used in \cite{ong2005terrain} and \cite{saunders2006realistic}, a user-provided sketch is used to generate a first 2D outline (see figure \ref{fig:tag15}). After that a hight map is generated from that data. The resulting terrain can then be modified further and be refined by the user with control points located at the terrain surface, interactively. With this two-phase approach the user keeps full control of basically all the input parameters, while still generating a completely new terrain more or less automatically.

Another technique is even capable of generating completely new terrain without any user-samples at all (see figures \ref{fig:tag19} and \ref{fig:tag21}): In the approach presented in \cite{frade2009breeding},\cite{frade2010evolution1},\cite{frade2010evolution2} and \cite{rodrigues2010development}, a stochastic noise algorithm is used to generate base hight-maps. After that a special hight function is applied to each vertex, working with a tree of operands. An obvious problem here is the fact, that it can not be guaranteed that the resulting terrain meets als requirements and is completely player-traversable (accessible with the player's avatar). Since this is quite an important and therefore critical criteria for most game-applications an accessibility measurement was introduced in later implementations of this algorithm. That way it's possible to generate usable maps for a variety of games, where the player can walk all across the surface of the terrain.

Both of these techniques have the big advantage of exploring the solution space quite well \cite{raffe2012survey}. But since no sample-terrains are used which already fit the required constraints, is has to be assured that the resulting terrain is usable with another process (either automatically or by the user).

\begin{figure}[htb]
	\centering
	\includegraphics[width=\linewidth]{RZL12/sjrjsrtr6zsr6z.jpg}
	\caption{Example of results with the algorithm from \cite{raffe2011evolving}}
	\label{fig:tag25}
\end{figure}

\subsection{Performance}
As all different approaches have their own up- and downsides, it is hard to objectively compare them. The decision which algorithms to use will more likely depend of the requirements for the terrain-generation process, the number and type of constraints necessary, and the level of user-interaction desired during the process. As far as the performance of the mentioned techniques go: Unfortunately neither in the reviewed state-of-the-art report itself, nor in the underlying papers presented there, any significant, comparable performance measurements which are usable for our purpose here can be found. But since all presented techniques solely rely on working with hight field data, it can be assumed that all mentioned approaches and algorithms are suitable for realtime interaction.