\section{Introduction}
Traditional content creation requires a vast part of development recourses. Therefore it is only plausible to look for more sophisticated solutions, that allow for faster, more accurate and physically plausible terrain generation and deformation. Procedural terrain generation has been a part research for a long time. It often consists of different pseudo-random noise functions of different frequencies, that are overlapped. This way simple non homogeneous landscapes with unsteady terrain elevations can be created with ease. Such approaches often lack major landmark features like rivers or big mountains, that, despite their natural occurrences, stand out of the surrounding landscape. But not only the generation of such terrains is a complex process, the simulation of effects on such terrains is an open field of study. The influence of fluid on a terrain is such a field. The term erosion describes the process of dissolving material and transport it from one location to another location, causing sediments. Such simulations heavily rely on physically correct flow simulations and a good erosion model, as well as stable data structures. 

Furthermore this report presents several solutions for more physically correct erosion simulations based on hydrology and heat transfer, fluid dynamics and particle movement, as well as tectonic plate movement. While the exact formulas  and implementation techniques will not be part of this report, we'll try to illustrate the use of every technique as well as we want to discuss their specific advantages and disadvantages.


\section{Taxonomy}

The variety of techniques and approaches is very wide in this research field. Therefore we will present several methods separately and leave the evaluation, discussion and conclusion to the very end of this report. As an introduction to this topic we discuss an approach by Benes et. al \cite{CGF:CGF12530}. This paper presents a novel approach on datastructures to create terrains as combinations of feature landmarks. They describe a terrain as a set of landmarks, that are combined using certain operators. this data structure is used in different papers throughout this report as well, since it provides a fast way of combining artifical and generated content in one terrain. 

The next paper \cite{Genevaux:2013:TGU:2461912.2461996} deals with large scale generation of river networks. A graph is created and iterativle grown to create a plausible river network. From this graph river primitives are generated and feeded as input into the data structure mentioned above. 

Cordonnier et. al. \cite{cordonnier2016large}  deal with river graphs as well, but on a more geological level. They use an approach of tectonic uplift to iteratively elevate nodes of a graph to simulate tectonic uplift. This graph changes its shape, in order to compensate for changes of flow direction or lake overflow. This approach simulates the process of tectonic uplift in combination with fluvial erosion to create large-scale terrains. Erosion can be simulated on such large scales, as well as granular scales. 

This paper \cite{Neidhold:2005:IPB:2381356.2381361} explores an approach of simulating fluvial erosion using real fluid simulations. The system accounts for dissolving of soil in water, as well as deposition of dissolved material. Several approaches are using fluid simulations. 

In this paper \cite{rungjiratananon2008real}the authors present a technique to of real-time sand-water interaction using a particle-based approach. Another paper that uses particle systems was written by Krivstof et. al. \cite{krivstof2009hydraulic}. They present two approaches on different fluid particle systems, that interact with the environment producing erosion in real time. 
Again Benes et. al \cite{benes2001layered} present a paper on this matter. Since in many fields full 3D voxel representation would be a waste of resources, they present an efficient and effective data structure that uses mutiple layers to represent a terrain. 
Another field of study for complex systems is heat transfer simulations for modeling realistic witner scenarios\cite{marechal2010heat}. The authors describe a complete physically based system, including dynamic thermal changes, causing ice and snow. 
Lastly, we present a paper \cite{raffe2012survey}, that gives an overview over evolutionary algorithms used in terrain generation. 


