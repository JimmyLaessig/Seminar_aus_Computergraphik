\section{Introduction}

Terrain modelling  and representation is a elaborated topic in computer graphics and visualization. Traditional content creation requires a vast part of development recourses. Therefore it is only plausible to look for more sophisticated solutions, that allow for faster, more accurate and physically plausible terrain generation and deformation. Procedural terrain generation has been a part research for a long time. It often consists of different pseudo-random noise functions of different frequencies, that are overlapped. This way simple non homogeneous landscapes with unsteady terrain elevations can be created with ease. This approaches often lack major landmark features like rivers or big mountains, that, despite their natural occurrences, stand out of the surrounding landscape. But not only the generation of such terrains is a complex process, the simulation of effects on such terrains is an open field of study. The influence of fluid on a terrain is such a field. The term erosion describes the process of dissolving material and transport it from one location to another location, causing sediments. Such simulations heavily rely on physically correct flow simulations and a good erosion model, as well as stable data structures. 

Furthermore this report presents several solution for more physically correct erosion simulations based on hydrology and heat transfer, fluid dynamics and particle movement, as well as tectonic plate movement. While the exact formulas  and implementation techniques will not be part of this report, we'll try to illustrate the use of every technique as well as we want to discuss their specific advantages and disadvantages.

The variety of techniques and approaches is very wide in this field of research and development. Therefore we will briefly present every method separately and leave the evaluation, discussion and conclusion to the very end of this report.


