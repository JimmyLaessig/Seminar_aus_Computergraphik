\section{Introduction}
Terrain modeling and representation is a elaborated topic in computer graphics and visualization. Traditional content creation requires a vast part of development recources. Therefore it is only plausible to look for more sophisticated solutions, that allow faster, more accurate and physically plausible terrain generation and representation. This report elaborates a handful of different solutions, that approach several different tasks in terrain generation. 
We will highlight several techniques on terrain modeling, datastructures and rendering approaches. 
Furthermore this report presents several solution for more physically correct erosion simulations based on hydrology and heat transfer, fluid dynamics and particle movement, as well as tectonic plate movement. While the exact formulas  and implementation techniques will not be part of this report, we'll try to illustrate the use of every technique as well as we want to discuss their specific advantages and disadvantages.

The variety of techniques and approaches is very wide in this field of research and development. Therefore we will briefly present every method separately and leave the evaluation, discussion and conclusion to the very end of this report.


%%modern terrain representations that exceed in both functionality and practicality over the traditional height field approach

%%Geologically spoken rivernetworks are created by an interplay of water and the terrain. Waterflow adpats to the terrain, as well as the terrain gets adapted to the waterflow eventually. We will talk about the influence of water 